\documentclass[article]{jss}

\author{David Meyer\\Wirtschaftsuniversit\"at Wien \And 
        Fridolin Wild\\Wirtschaftsuniversit\"at Wien}

\title{\pkg{lsa}: an \proglang{R} Package for Latent Semantic Analysis}

\Plainauthor{David Meyer, Fridolin Wild} 
\Plaintitle{lsa: an R Package for Latent Semantic Analysis}
\Shorttitle{Latent Semantic Analysis}

\Abstract{
  Latent semantic analysis (LSA) has continuously stirred scientific interest
  since its invention in the early 1990ies. LSA combines the classical 
  vector space model -- well known in computational linguistics -- with a 
  singular value decomposition (SVD), a two-mode factor analysis. Thereby, 
  bag-of-words representations of texts can be mapped into a modified vector
  space that is assumed to reflect semantic structure. LSA can be 
  applied for various purposes, e.g. to automatically evaluate free text 
  responses typical for educational assessments.
 
  In this paper we describe the \pkg{lsa} package for the language and
  environment \proglang{R} and illustrate its proper use through 
  examples from the area of automated essay scoring. The package includes
  several sets of functions supporting the LSA process from construction of 
  document-term-matrices, application of text preprocessing methods and weighting 
  schemes, selection of the desired number of factors, and calculation of 
  similarities within the newly created, latent semantic space.

}

\Keywords{lsa, latent semantic analysis, text technology, computer linguistics, \proglang{R}}
\Plainkeywords{lsa, latent semantic analysis, text technology, computer linguistics, R}

%% publication information
%% \Volume{13}
%% \Issue{9}
%% \Month{September}
%% \Year{2004}
%% \Submitdate{2004-09-29}
%% \Acceptdate{2004-09-29}

\Address{
  David Meyer, Fridolin Wild\\
  Abteilung f\"ur Wirtschaftsinformatik und neue Medien\\
  Wirtschaftsuniversit\"at Wien\\
  1090 Wien, Austria\\
  E-mail: \email{vorname.nachname@wu-wien.ac.at}
}

%% for those who use Sweave please include the following line (with % symbols):
%% need no \usepackage{Sweave.sty}

\begin{document}

\section[statistical technique]{Introduction}

  Derived from latent semantic indexing, LSA is intended to enable the 
  analysis of the semantic structure of texts. The basic idea behind LSA 
  is that the collocation of terms of a given document-term-space reflects 
  a higher-order -- latent semantic -- structure, which is obscured by 
  word usage (e.g. by synonyms or ambiguities). By using conceptual indices 
  that are derived statistically via a truncated singular value decomposition, 
  this variability problem is believed to be overcome, cf. \cite{landauer:1990}.
  
  In a typical LSA process, first a document-term-matrix is constructed 
  from a given text base of \begin{math}n\end{math} documents containing 
  \begin{math}m\end{math} terms (see \begin{math}M\end{math} in figure~\ref{fig:process}). 
  This matrix \begin{math}M\end{math} of the size \begin{math}m \times n\end{math} 
  is then resolved by the singular value decomposition into the term vector 
  matrix \begin{math}T\end{math} (constituting the right singular vectors), 
  the document vector matrix \begin{math}D\end{math} (constituting the right 
  singular vectors) being both orthonormal and the diagonal matrix 
  \begin{math}S\end{math}. Multiplying the truncated matrices 
  \begin{math}T_{k}\end{math}, \begin{math}S_{k}\end{math} and 
  \begin{math}D_{k}\end{math} results in a new matrix \begin{math}M_{k}\end{math} 
  which is the least-squares best fit approximation of 
  \begin{math}M\end{math} with \begin{math}k\end{math} singular values.

  \begin{figure}
     \centering
     \label{fig:process}
     \includegraphics[width=140mm]{lsa_process.jpg}
     \caption{Singular Value Decomposition (original left, truncated right).}
  \end{figure}

\section[overview]{Latent Semantic Analysis}

Folding-In, Steps in the Process

\section[explaining code]{The \pkg{lsa} Package}

a section explaining the code

\section[examples]{Some Examples of Use}

a section with examples.
taken from essay scoring.

\section[conclusions]{Conclusions}

Future plans.

\section[acknowledgements]{Acknowledgements}

Christina Stahl, Gerald Stermsek, Gustaf Neumann, Yoseba Penya, Kurt Hornik, Stefan Sobernig.
Art Graeser, Max Louwerse, Roberto Turrin, Brian Ripley, Milos Kravcik, Goto...

\bibliography{lsa}

\end{document}
